% Created 2021-04-01 Thu 10:27
% Intended LaTeX compiler: pdflatex
\documentclass{amsart}
\usepackage{enumerate}
\usepackage{amsthm,amscd,amssymb,verbatim,epsf,amsmath,amsfonts,mathrsfs,graphicx}
\usepackage[linktocpage,colorlinks=true,linkcolor=blue,citecolor=blue]{hyperref}
\usepackage{lineno}
% The bibliography file
% Theorem Environment.
\newtheorem{thm}{Theorem}
\newtheorem{lem}{Lemma}
\newtheorem{prop}{Proposition}
\newtheorem{prob}{Problem}
\newtheorem{cor}{Corollary}
\newtheorem{mydef}{Definition}
\newtheorem{conj}{Conjecture}

% Here some definitions of operators
% Trace operator.
\DeclareMathOperator{\Tr}{Tr}
% Divergence Operator.
\DeclareMathOperator{\diver}{div}
% Index of an  Operator.
\DeclareMathOperator{\ind}{ind}
% Diameter.
\DeclareMathOperator{\diamm}{diam}
% Inverse of hyperbolic tan
\DeclareMathOperator\artanh{artanh}
% Distance
\DeclareMathOperator{\dist}{dist}
% Scalar Curvature
\DeclareMathOperator{\R}{dist}
% Ricci Curvature
\DeclareMathOperator{\Ric}{Ric}

\usepackage[utf8]{inputenc}
\usepackage[T1]{fontenc}
\usepackage{graphicx}
\usepackage{grffile}
\usepackage{longtable}
\usepackage{wrapfig}
\usepackage{rotating}
\usepackage[normalem]{ulem}
\usepackage{amsmath}
\usepackage{textcomp}
\usepackage{amssymb}
\usepackage{capt-of}
\usepackage{hyperref}
\author{Daniel Ballesteros Chávez}
\date{\textit{<2021-03-12 Fri>}}
\title{Lecture 4. Introduction to Complex Analysis}
\hypersetup{
 pdfauthor={Daniel Ballesteros Chávez},
 pdftitle={Lecture 4. Introduction to Complex Analysis},
 pdfkeywords={},
 pdfsubject={},
 pdfcreator={Emacs 26.1 (Org mode 9.3.6)}, 
 pdflang={English}}
\begin{document}

\maketitle
\tableofcontents





\section{Sequences and limits}
\label{sec:org2fecbf5}

\begin{mydef}
A sequence in \(\mathbb{C}\) is a function \(\varphi: \mathbb{N}\to \mathbb{C}\). We use the notation
\(z_n = \varphi(n)\) and \(\{z_n\}_{n=1}^{\infty}\) for the image \(\varphi(\mathbb{N})\).
\end{mydef}


\begin{mydef}
Let \(\{z_n\}_{n=1}^{\infty}\) be a sequence in \(\mathbb{C}\). We say that the sequence converges to \(z_0\in \mathbb{C}\)
if for every \(\epsilon >0\) there exists \(n_0\in \mathbb{N}\), such that for every \(n\geq n_0\) it holds that \(|z_n - z_0| < \epsilon\). In this case
\(z_0\) is called the limit of the sequence \(\{z_n\}_{n=1}^{\infty}\) and we write
\[ \lim_{n\to\infty} z_n = z_0. \]
\end{mydef}

\begin{prop}
If a sequence converges to a limit, this is unique.
\end{prop}
\begin{proof}
Supose that the sequence \(\{z_n\}_{n=1}^{\infty}\subset \mathbb{C}\) has two limits \(z_0\) and \(w_0\). Then for any \(\epsilon >0\) choose \(n\in \mathbb{C}\) such that for any \(n\geq n_0\), 
we have simultaneously \[ |z_n  - z_0| < \frac{\epsilon}{2}, \mbox{ and } |z_n - w_0 | < \frac{\epsilon}{2}. \] 
Then
\[ | z_0 - w_0 | = |z_0 - z_n + z_n - w_0 | \leq | z_n - z_0 | + | z_n - w_0| < \frac{\epsilon}{2} + \frac{\epsilon}{2} = \epsilon. \]
Since \(\epsilon >0\) is arbtrary, this happens only if |z\textsubscript{0} - w\textsubscript{0}|  = 0. Then \(z_0 = w_0\).
\end{proof}


\begin{prop}
Let \(\{z_n\}_{n=1}^{\infty}\) be a sequence in \(\mathbb{C}\), with \(z_n = x_n + i y_n\), and \(z_0 = x_0 + iy_0\). Then 
the sequence converges to \(z_0\) if and only if the sequences of real numbers \(\{x_n\}_{n=1}^{\infty}\) and \(\{y_n\}_{n=1}^{\infty}\) converge respectively to \(x_0\) and \(y_0\). Equivalently
\[ \lim_{n\to\infty}z_n = z_0 \iff \lim_{n\to\infty}x_n = x_0 \mbox{ and } \lim_{n\to\infty}y_n = y_0. \]
\end{prop}
\begin{proof}
Let \(\epsilon >0\). Then there exists \(n_0 \in \mathbb{N}\), such that for any \(n\geq n_0\) the following holds:
\[ |x_n - x_0 | = | \mbox{Re}(z_n - z_0) | \leq |z_n - z_0 | < \epsilon. \]
This shows that \(\lim_{n\to \infty} x_n = x_0\). The second part follows from a similar argument on the imaginary part of the sequence.
\end{proof}

\begin{prop}
Every convergent sequence \(\{z_n\}_{n=1}^{\infty} \subset \mathbb{C}\) is bounded, i.e., there exists
a positive real number \(M>0\), such that \(|z_n| \leq M\) for all \(n\in\mathbb{N}\).
\end{prop}
\begin{proof}
Let \(\epsilon = 1\). By convergence, there exists \(n_0\in \mathbb{N}\) such that whenever \(k\geq n_0\) it holds that 
\[ |z_k - z_0 | < 1. \]
Now consider \(M = \max\{ |z_1| + 1, |z_2| + 1, \ldots, |z_{k-1}| + 1, |z_0| + 1\}\). Then it is clear that \[ |z_n| \leq M, \mbox{ for all } n \in \mathbb{N}. \]
\end{proof}

\textbf{Remark} The converse of last proposition is false. Discuss an example.

\begin{thm}
(Bolzano-Weierstrass) Every bounded sequence in \(\mathbb{C}\) has a convergent sub-sequence.
\end{thm}
\begin{proof}
The proof relies on the same result known for sequences of Real numbers.

Let \(\{z_n\}_{n=1}^{\infty}\) a bounded sequence where \(z_n = x_n + iy_n\). Since the sequence is bounded we note  that 
\[ |x_n | \leq |z_n| \leq M, \quad n\in\mathbb{N}. \]
Then the sequence of real numbers \(\{x_n\}_{n=1}^{\infty}\) is bounded. There exists a convergent subsequence \(\left\{x_{n_k}\right\}_{k=1}^{\infty}\) with limit \(x_0\in \mathbb{R}\).
Now consider the corresponding subsequence of complex numbers \(\left\{z_{n_k}\right\}_{k=1}^{\infty}\), where \(z_{n_k} = x_{n_k} + i y_{n_k}\). Since the sequence of complex numbers is bounded,
we conclude that \(\left\{y_{n_k}\right\}_{k=1}^{\infty}\) is also a bounded sequence of Real numbers. Then there exists a convergent subsequence \(\left\{y_{n_{k_j}}\right\}_{j=1}^{\infty}\) wiht limit \(y_0\).
Recall that any subsequence of a convergent sequence is also convergent to the same limit, and conclude that \(\left\{x_{n_{k_j}}\right\}_{j=1}^{\infty}\) has limit \(x_0\). Then 
\[ \lim_{j\to \infty} z_{n_{k_j}} = \lim_{j\to \infty} \left(x_{n_{k_j}} + i y_{n_{k_j}}\right)  = x_0 + iy \]
is a convergent subsequence of  \(\{z_n\}_{n=1}^{\infty}\).
\end{proof}

\begin{mydef}
A sequence \(\{z_n\}_{n=1}^{\infty}\) is called Cauchy sequence if for every \(\epsilon >0\)
there exists \(n_0\in\mathbb{N}\) such that, if \(n,m \geq n_0\) then \(|z_n - z_m| <\epsilon\).
\end{mydef}


\begin{thm}
A sequence \(\{z_n\}_{n=1}^{\infty}\) converges if and only if \(\{z_n\}_{n=1}^{\infty}\) is a Cauchy sequence.
\end{thm}
\begin{proof}
By writing \(z_n = x_n + i y_n\) and from the inequalities
\[ |x_n - x_m | \leq |z_n - z_m|, \]
and
\[ |y_n - y_m | \leq |z_n - z_m|, \]
it follows that The sequence  \(\{z_n\}_{n=1}^{\infty}\) of complex numbers is Cauchy if and only if  \(\{x_n\}_{n=1}^{\infty}\) and \(\{y_n\}_{n=1}^{\infty}\) are Cauchy sequences of Real numbers if and only if  the sequences of real numbers \(\{x_n\}_{n=1}^{\infty}\) and \(\{y_n\}_{n=1}^{\infty}\) are convergent if and only if \(\{z_n\}_{n=1}^{\infty}\) is a convergent sequence.
\end{proof}


\begin{prop}
Let \(\{z_n\}_{n=1}^{\infty}\) and \(\{w_n\}_{n=1}^{\infty}\) two convergent sequences, with \(\lim_{n\to\infty}z_n = z_0\) and \(\lim_{n\to\infty}w_n = w_0\). Then the following identities hold:

\begin{enumerate}
\item \(\displaystyle\lim_{n\to\infty}(z_n + w_n) = z_0 + w_0\),
\item \(\displaystyle\lim_{n\to\infty}(z_n \, w_n) = z_0 \, w_0\),
\item \(\displaystyle\lim_{n\to\infty}|z_n| = |z_0|\),
\item If \(w_n \neq 0\) for all \(n\in \mathbb{N}\), then \(\displaystyle\lim_{n\to\infty}\left(\dfrac{z_n}{w_n}\right) = \dfrac{z_0}{w_0}\).
\end{enumerate}
\end{prop}
\begin{proof}
Proof of 1. It follows from the triangle inequality and the definition. Let \(\epsilon > 0\), then there exists \(n_0 \in \mathbb{N}\) such that for any \(n\geq n_0\) it simultaneously holds \(|z_n - z_0| < \frac{\epsilon}{2}\) and \(|w_n - w_0| < \frac{\epsilon}{2}\), then 
\[ |(z_n + w_n) - (z_0 + w_0) | = |(z_n - z_0) + (w_n - w_0) | < \frac{\epsilon}{2} + \frac{\epsilon}{2} = \epsilon. \]
That is \(\displaystyle\lim_{n\to\infty}(z_n + w_n) = z_0 + w_0\).

Proof of 2. It follows also from the definition, the triangle inequality and the fact that every convergent sequence is bounded. Since \(\{z_n\}_{n=1}^{\infty}\) is convergent, there is \(M_1>0\) such that \(|z_n| < M_1\) for all \(n\in\mathbb{N}\) and the same for \(\{w_n\}_{n=1}^{\infty}\), there is \(M_2 >0\) such that \(|w_n| < M_1\) for all \(n\in\mathbb{N}\). Take \(M = \max\{M_1,M_2\}\), and then \(M>0\) is a bound for both sequences.  Let \(\epsilon > 0\), then there exists \(n_0 \in \mathbb{N}\) such that for any \(n\geq n_0\) it simultaneously holds \(|z_n - z_0| < \frac{\epsilon}{2M}\) and \(|w_n - w_0| <  \frac{\epsilon}{2M}\). Then 
\begin{equation*} \begin{split} |z_n w_n - z_0 w_0 | &= | z_n w_n - z_n w_0 + z_n w_0 - z_0 w_0 | \\ & \leq | z_n w_n - z_n w_0 | + |z_n w_0 - z_0 w_0 | \\ &  = |z_n| | w_n - w_0| + |w_0| |z_n - z_0|  \\ & \leq M | w_n - w_0| + M |z_n - z_0| \\ & < \frac{\epsilon}{2} +  \frac{\epsilon}{2} = \epsilon. \end{split} \end{equation*}
That is \(\displaystyle\lim_{n\to\infty}(z_n \, w_n) = z_0 \, w_0\).

Proof of 3. It follows from a version of the triangle inequality. Let \(\epsilon > 0\), then there exists \(n_0 \in \mathbb{N}\) such that for any \(n\geq n_0\) it holds \(|z_n - z_0| < \epsilon\). Note then that
\[ |\, |z_n| - |z_0| \, | \leq |z_n - z_0| < \epsilon. \]
That is \(\displaystyle\lim_{n\to\infty}|z_n| = |z_0|\).

Proof of 4. Let \(\epsilon_1 = \frac{|w_0|}{2}\), then there exists \(n_0 \in \mathbb{N}\) such that for any \(n\geq n_0\) it holds \(|w_n - w_0| < \frac{|w_0|}{2}\). First note that
\[ \left| \frac{1}{w_n} - \frac{1}{w_0} \right| = |\frac{ w_0 - w_n}{w_n w_0} | = \frac{ |w_0 - w_n|}{|w_n |\, |w_0|}. \]

On the other hand by the triangle inequality
\[ |w_0| - |w_n| \leq \left|\, |w_0| - |w_n|\, \right| \leq |w_0 - w_n | < \frac{|w_0|}{2}, \]
this implies
\[ \frac{|w_0|}{2} \leq |w_n|, \]
or equivalently
\[ \frac{1}{|w_n|} \leq \frac{2}{|w_0|}. \]
Then we get
\[ \left| \frac{1}{w_n} - \frac{1}{w_0} \right| \leq \frac{ |w_0 - w_n|}{|w_n |\, |w_0|} \leq \frac{2 |w_0 - w_n|}{ |w_0|^2}. \]

For any \(\epsilon >0\) define \(\epsilon_2 = \frac{|w_0|^2\epsilon}{2}\), then there exists \(m_0 \in \mathbb{N}\) such that for any \(n\geq m_0\) it holds \(|w_n - w_0| < \frac{|w_0|^2\epsilon}{2}\).
Take \(N_0 = \max\{n_0, m_0\}\). Then we can improve our last inequality, for any \(n \geq N_0\) it holds
\[ \left| \frac{1}{w_n} - \frac{1}{w_0} \right| \leq \frac{ |w_0 - w_n|}{|w_n |\, |w_0|} \leq \frac{2 |w_0 - w_n|}{ |w_0|^2} < \epsilon. \]
This shows that  \(\displaystyle\lim_{n\to\infty}\frac{1}{w_n} = \frac{1}{w_0}\).
Finally applying (2) of this proposition we have
\[ \displaystyle\lim_{n\to\infty}\frac{z_0}{w_n} = \lim_{n\to\infty} z_n\frac{1}{w_n} = \lim_{n\to\infty} z_n \lim_{n\to\infty}\frac{1}{w_n} = \frac{z_0}{w_0} \]
\end{proof}

Recall from real analysis, that every monotone and bounded sequence of real numbers is convergent.
\begin{itemize}
\item If the sequence \(\{a_n\}_{n=1}^{\infty}\) is increasing then \(\lim_{n\to\infty} a_n = \sup\{a_n| n\in \mathbb{N}\}\).
\item If the sequence \(\{a_n\}_{n=1}^{\infty}\) is decreasing then \(\lim_{n\to\infty} a_n = \inf\{a_n| n\in \mathbb{N}\}\).
\end{itemize}


\begin{mydef}
We say that \(\lim_{n\to\infty} z_n = \infty\) if \(\lim_{n\to \infty}|z_n| = \infty\). In other words,
if for every \(M>0\) there exists \(n_0\in\mathbb{N}\) such that for all \(n\geq n_0\) then \(|z_n| > M\).
\end{mydef}


\begin{mydef}
Let \(\{z_n\}_{n=1}^{\infty} \subset \mathbb{C}\) be a sequence. A complex number \(w_0\in\mathbb{C}\) is 
called a limit point of the sequence \(\{z_n\}_{n=1}^{\infty}\) if there exists a subsequence
\(\{z_{n_k}\}_{k=1}^{\infty}\) of \(\{z_n\}_{n=1}^{\infty}\) such that
\[ \lim_{k\to\infty}z_{n_k} = w_0. \]
\end{mydef}


\section{Continuity}
\label{sec:org2444c93}
\begin{mydef}
The function \(f(z)\) is said to have the limit \(w_0\in \mathbb{C}\) as \(z\) tends to \(a\in\mathbb{C}\),  if for every \(\epsilon >0\) there exists a number \(\delta = \delta(\epsilon) >0\) with the property that if \(|z - a| < \delta\) then it holds \(|f(z) - w_0| <\epsilon\), and in that case we write \[ \lim_{z\to a} f(z) = w_0 . \]
\end{mydef}


\begin{mydef}
The function \(f(z)\) is said to be continuous at \(z_0\) if for if for every \(\epsilon >0\) there exists a number \(\delta = \delta(\epsilon) >0\) with the property that if \(|z - a| < \delta\) then it holds \(|f(z) - f(a)| <\epsilon\), and in that case we write \[ \lim_{z\to a} f(z) = f(a) . \]
\end{mydef}

\textbf{Examples}
\begin{itemize}
\item Let \(f:\mathbb{C}\setminus\{1\} \to \mathbb{C}\) the function given by \[f(z) =\dfrac{z^2 - 4}{z - 2}, \] then \(\lim_{z\to 2} f(z) = 4\).

\item Let \(f:\mathbb{C} \to \mathbb{C}\) be the conjugate function \(f(z) = \bar{z}\). Then \(f(z)\) is continuous at every \(z_0\in \mathbb{C}\). To show this, first note that \(|z| = |\bar{z}|\). Let \(\epsilon >0\) and choose \(0< \delta = \epsilon\). Then, for any \(z\in \mathbb{C}\) such that \(|z-z_0|\) we can estimate \[ |f(z) - f(z_0)| = |\bar{z} - \bar{z_0}|=|\overline{z - z_0}| = |z- z_0| < \delta = \epsilon.\].
\end{itemize}

\begin{prop}
Let \(U \subset \mathbb{C}\), \(f: U \to \mathbb{C}\) and \(a\in U\). Then the following three statements are equivalent:
\begin{enumerate}
\item \(f\) is continuous at \(a\).
\item For every \(\epsilon >0\) there exists a number  \(\delta >0\) such that \(f\left( B_{\delta}(a) \cap U\right) \subseteq B_{\epsilon}(f(a))\).
\item For every sequence \(\{z_n\}_{n=1}^{\infty}\subseteq U\) such that \(\displaystyle\lim_{n\to \infty} z_n = a\), we have \[\displaystyle\lim_{n\to \infty} f(z_n) = f(a).\]
\end{enumerate}
\end{prop}
\begin{proof}
(1 \(\Rightarrow\) 2) Let \(\epsilon >0\), then there is a number \(\delta >0\)  such that for every \(z\in U\) with \(|z-a| < \delta\) then \(|f(z) - f(a)| < \epsilon\). This implies that if \(z\in B_{\delta}(a) \cap U\)  then \(f(B_{\delta}(a) \cap U)\subseteq B_{\epsilon}(f(a))\).

(2 \(\Rightarrow\) 1) Let \(\epsilon >0\), then there is a number \(\delta >0\)  such that  \(f(B_{\delta}(a) \cap U)\subseteq B_{\epsilon}(f(a))\). This means that for every  \(z\in B_{\delta}(a) \cap U\) we have \(f(z)\in B_{\epsilon} (f(a))\). This means that for every  \(z\in U\) with  \(|z-a| < \delta\) it holds \(|f(z) - f(a)| < \epsilon\). 

(1 \(\Rightarrow\) 3) Let \(f(z)\) be continuous at \(a\in U\) and \(\{z_n\}_{n=1}^{\infty}\subseteq U\) with limit \(a\). For any \(\epsilon >0\) there exists a number \(\delta >0\) such that for every \(z\in U\) with \(|z-a|<\delta\) it holds \(|f(z)-f(a)|  < \epsilon\). For that \(\delta >0\) there is a \(n_0\in \mathbb{N}\) such that for any \(n\geq n_0\) we have \(|z_n - a| \leq \delta\). Then by continuity this implies that \(|f(z_n) - f(a) | < \epsilon\). This shows that \(\lim_{n\to \infty} f(z_n) = f(a)\).

(3 \(\Rightarrow\) 1) Let \(a\in U\) and assume that for any \(\{z_n\}_{n=1}^{\infty}\subseteq U\) with limit \(a\) we have \(\lim_{n\to \infty} f(z_n) = f(a)\). Suppose on the contrary, that \(f\) is not continuous at \(a\). Then there exists a positive \(\epsilon >0\) such that for every \(\delta >0\) there is a point \(z\in U\) with \(|z -a| < \delta\) but \(|f(z) - f(a)| \geq \epsilon\). Note that this implies that for any \(n\in\mathbb{N}\) we can choose \(\delta = \frac{1}{n}\) and  \(z_n\in U\) such that \(|z_n - a| < \frac{1}{n}\), but  \(|f(z_n) - f(a)| \geq \epsilon\). In this way we have constructed a sequence \(\{z_n\}_{n=1}^{\infty}\) such that \(\lim_{n\to\infty}z_n = a\) but \(\lim_{n\to\infty}f(z_n) \neq f(a)\) which is a contradiction to our initial assumptions.

(1 \(\Rightarrow\) 3) Let \(f(z)\) be continuous at \(a\). Consider any sequence \(\{z_n\}_{n=1}^{\infty}\subseteq U\) such that \(\lim_{n\to \infty} z_n = a\). Then for \(\epsilon >0\) there is a number \(\delta >0\) such that for all \(z\in U\) with \(|z-a|<\delta\) we have \(|f(z) - f(a)| < \epsilon\). Also for such \(\delta >0\), there exists \(n_0\in \mathbb{N}\) such that for all \(n\geq n_0\) we have \(|z_n - a| < \delta\), which then implies \(|f(z_n) - f(a)| < \epsilon\).
\end{proof}


\begin{thm}
Let \(U\subseteq\mathbb{C}\) and \(f,g:U\to \mathbb{C}\) be two continuous functions  at \(a\in U\). Then \(f\pm g\), \(f\cdot g\) are continuous in \(a\). If \(g(a) \neq 0\) then \(\dfrac{f}{g}\) is also continuous at \(a\).
\end{thm}


\begin{thm}
Let \(U_1, U_2 \subseteq \mathbb{C}\), and \(f:U_1\to \mathbb{C}\), \(g:U_2\to \mathbb{C}\), such that \(f(U_1)\subseteq U_2\). If \(f\) is continuous at \(z_0\in U_1\) and g is continuous at \(w_0 = f(z_0)\in U_2\), then the composition \(g\circ f\) is continuous at \(z_0\).
\end{thm}
\begin{proof}
Consider any sequence \(\{z_n\}_{n=1}^{\infty}\subseteq U_1\) such that \(\lim_{n\to \infty}z_n = z_0\). Then define \(w_n = f(z_n) \in U_2\). Since \(f\) is continuous at \(z_0\), then \(\lim_{n\to\infty}f(z_n) = \lim_{n\to\infty}w_n = w_0 = f(z_0)\). Since \(g\) is continuous at \(w_0\) then \(\lim_{n\to \infty}g(w_n) = g(w_0)\). Note that on the left hand side \(g(w_n) = g(f(z_n)) = (g\circ f)(z_n)\), and on the right hand side \(g(w_0) = g(f(z_0)) = (g\circ f)(z_0)\). This shows that \(\lim_{n\to\infty}(g\circ f)(z_n) = (g\circ f)(z_0)\).
\end{proof}


\begin{mydef}
Let \(U\subseteq \mathbb{C}\) and consider a function \(f:U\to\mathbb{C}\). \(f\) is called uniformly continuous if for every \(\epsilon >0\) there exists \(\delta >0\) such that for any \(z,w\in U\) such that \(|z-w| < \delta\) then we have \(|f(z) - f(w) | < \epsilon\).
\end{mydef}

\textbf{Example}. Consider \(f: A\to \mathbb{C}\) , \(f(z) = z^2\) in the following cases:
\begin{itemize}
\item \(A = \{z\in \mathbb{C} \, |\, |z| \leq 1\}\).
\end{itemize}

In this case \(f\) is unifrmly continuous in \(A\). Let \(\epsilon>0\) and take \(\delta = \epsilon/2\). Then, for any \(z,w\in A\) such that \(|z-w| < \delta\) we have
\[ |f(z) - f(w) | = |z^2 - w^2| = |z-w||z+w| \leq |z-w| (|z|+|w|) \leq 2 |z-w| < \epsilon. \]

\begin{itemize}
\item \(A = \mathbb{C}\).
\end{itemize}

In this case \(f\) is not uniformly continuous. Take \(\epsilon = 1\). For every \(\delta >0\) there exists \(n\in \mathbb{N}\) such that \(n\delta > 1\). Now consider \(z= n\) and \(w = n + \delta/2\). Note that we have
\(|z - w| = \delta/2 < \delta\), but
\[ |f(z) - f(w) | = |n^2 - \left(n + \frac{\delta}{2}\right)^2| =  n\delta + \frac{\delta^2}{4} > n\delta > 1 = \epsilon. \]

\section{Basic topology of \(\mathbb{C}\)}
\label{sec:orge7247dc}

The open and closed disks (balls) already defined, are basic subsets that may be used to build a topological structure of the complex plane. A topology allows us to define several notions of continuity.

\[ B_r(z_0) = \{ z\in \mathbb{C} \, | \, |z-z_0| < r\}. \]
\[ \bar{B}_r(z_0) = \{ z\in \mathbb{C} \, | \, |z-z_0| \leq r\}. \]

\begin{mydef}
A set \(A\subseteq \mathbb{C}\) is called open if \(\forall z\in A\) there exists a real number \(r_z >0\) such that \(B_{r_z}(z)\subseteq A\).
\end{mydef}


\begin{thm}
The following sentences are true
\begin{enumerate}
\item The sets \(\mathbb{C}\), \(\emptyset\), \(B_r(z)\) ( for any \(r>0\) ) and any \(z\in \mathbb{C}\), are open sets.
\item If \(U_1, \ldots, U_n\) is a finite collection of open sets, then \(\cap_{k=1}^n U_{k}\) is an open set.
\item If \(\{U_{\alpha}\}_{\alpha\in I}\) is a family of open sets, then \(\cup_{\alpha\in I}U_{\alpha}\) is an open set.
\end{enumerate}
\end{thm}
\begin{proof}
\(\mathbb{C}\) is open. For any \(z\in \mathbb{C}\), take \(r=1\) and evidently \(B_{1}(z)\subset \mathbb{C}\).

\(\emptyset\) is open. If this wasn't true, there would exists \(z\in \emptyset\) such that for any \(r>0\), \(B_{r}(z)\nsubseteq \emptyset\). Then the statement is vacuously true.

\(B_{r}(z)\) is open. Let \(w\in B_{r}(z)\). Take \(\delta = r - |w-z|\), and note that \(\delta >0\). To show that \(B_{\delta}(w) \subset B_{r}(z)\) we need to show that any \(\xi \in B_{\delta}(w)\) is also in \(B_{r}(z)\). If \(\xi \in B_{\delta}(w)\) then \(|\xi - w| < \delta\). Since \(|\xi - z| = |\xi - w + w - z| \leq |\xi - w| + | w - z| < \delta + |w-z| = r - |w-z| + |w-z| = r\). Then \(\xi \in B_{r}(z)\).

\(\cap_{k=1}^n U_k\) is open if each \(U_i\) is open. Let \(z\in\cap_{k=1}^n U_k\). Then for each \(1\leq k \leq n\) there exists \(r_{k}>0\) such that \(B_{r_k}(z) \subseteq U_k\). Take \(r = \min\{r_1,\ldots,r_k\}\), and hence \(B_{r}(z)\subset U_k\) for every \(1\leq k \leq n\). Hence \(B_{r}(z)\subset \cap_{k=1}^n U_k\).

\(\cup_{\alpha \in I} U_{\alpha}\) is open if each \(U_{\alpha}\) is open. Let \(z\in\cup_{\alpha \in I} U_{\alpha}\). Then there exists \(\beta \in I\) such that \(z \in U_{\beta}\). Since \(U_{\beta}\) is open, there is \(r >0\) such that \(B_{r}(z)\subset U_{\beta} \subset \cup_{\alpha \in I} U_{\alpha}\).
\end{proof}


\begin{mydef}
A set \(G\subseteq \mathbb{C}\) is called closed if its complement \(\mathbb{C}\setminus G\), is an open set.
\end{mydef}

\begin{prop}
The following sentences are true
\begin{enumerate}
\item The sets \(\mathbb{C}\), \(\emptyset\), are closed sets.
\item If \(G_1, \ldots, G_n\) is a finite collection of closed sets, then \(\cup_{k=1}^n G_{k}\) is a closed set.
\item If \(\{G_{\alpha}\}_{\alpha\in I}\) is a family of closed sets, then \(\cap_{\alpha\in I}G_{\alpha}\) is a closed set.
\end{enumerate}
\end{prop}


\begin{mydef}
Let \(A\subset \mathbb{C}\). Then we define the following sets
\begin{itemize}
\item The interior of \(A\): \(\quad\mbox{Int}(A) = \cup \{ U \,|\, U\subset A, \mbox{ and } U \mbox{ is an open set} \}\).
\item The closure of \(A\): \(\quad\mbox{Cl}(A) = \cap \{ G \,|\, A \subset G, \mbox{ and } G \mbox{ is a closed set} \}\).
\item The boundary of \(A\): \(\quad\partial A = \mbox{Cl}(A) \cap \mbox{Cl}\left( \mathbb{C} \setminus A\right)\).
\end{itemize}
\end{mydef}


\begin{prop}
Let \(A,B\subseteq \mathbb{C}\). Then following statements are true
\begin{itemize}
\item \(\mbox{Int}(A)\subset A\).
\item \(A \subset \mbox{Cl}(A)\).
\item \(\partial A \subseteq \mbox{Cl}(A)\).
\item \(A\) is open if and only if \(A = \mbox{Int}(A)\).
\item \(A\) is closed if and only if \(A = \mbox{Cl}(A)\).
\item \(\mbox{Cl}(A\cup B) = \mbox{Cl}(A) \cup \mbox{Cl}(B)\).
\item \(\mbox{Cl}(A\cap B) \subseteq \mbox{Cl}(A) \cap \mbox{Cl}(B)\), but in general they are not equal.
\item \(\mbox{Int}(A\cap B) = \mbox{Int}(A) \cap \mbox{Int}(B)\).
\item \(\mbox{Int}(A\cup B) \supseteq \mbox{Int}(A) \cup \mbox{Int}(B)\), but in general they are not equal.
\item \(z_0\in \mbox{Int}(A) \Leftrightarrow \exists r>0, \mbox{ such that } B_{r}(z_0)\subseteq A\).
\item \(z_0\in \mbox{Cl}(A) \Leftrightarrow \forall r>0, \mbox{ it holds } B_{r}(z_0)\cap A \neq \emptyset\).
\item \(z_0\in \partial(A) \Leftrightarrow \forall r>0, \mbox{ it holds } B_{r}(z_0)\cap A \neq \emptyset \mbox{ and } B_{r}(z_0)\cap \left(\mathbb{C} \setminus A\right)  \neq \emptyset\).
\end{itemize}
\end{prop}

The topological structure of \(\mathbb{C}\) induces in a natural way a topological structure in any subset \(A\subset \mathbb{C}\).

\begin{mydef}
Let \(A\subset \mathbb{C}\). The set \(B\subseteq A\) is called open (closed) in \(A\), if there is an open (closed) set \(U\) of \(\mathbb{C}\) such that \(B = A \cap U\).
\end{mydef}

\begin{prop}
Let \(A\subseteq \mathbb{C}\). Then the following sentences are true
\begin{enumerate}
\item The sets \(A\), \(\emptyset\), are open sets and closed sets in \(A\).
\item If the set \(B\subseteq A\) is open in \(A\), then \(A\setminus B\) is closed in \(A\).
\item If the set \(B\subseteq A\) is closed in \(A\), then \(A\setminus B\) is open in \(A\).
\item If \(U_1, \ldots, U_n\) is a finite collection of open sets in \(A\), then \(\cap_{k=1}^n U_{k}\) is an open set in \(A\).
\item If \(\{U_{\alpha}\}_{\alpha\in I}\) is a family of open sets in \(A\), then \(\cup_{\alpha\in I}U_{\alpha}\) is an open set \(A\).
\item If \(G_1, \ldots, G_n\) is a finite collection of closed sets in \(A\), then \(\cup_{k=1}^n G_{k}\) is a closed set in \(A\).
\item If \(\{G_{\alpha}\}_{\alpha\in I}\) is a family of closed sets in \(A\), then \(\cap_{\alpha\in I}G_{\alpha}\) is a closed set in \(A\).
\end{enumerate}
\end{prop}

\begin{prop}
Let \(A\subset \mathbb{C}\). \(z_0\in \mbox{Cl}(A)\) if and only if there exists a sequence \(\{z_n\}_{n}^{\infty}\subset A\) such that \(\lim_{n\to\infty}z_n = z_0\).
\end{prop}
\begin{proof}
By the previous proposition, since \(z_0\in \mbox{Cl}(A)\), we have that for every \(n\in \mathbb{N}\) we can take \(r_n = \frac{1}{n} >0\), and it holds
\[ B_{r_n}(z_0)\cap A \neq \emptyset .\]
Then for each \(n\) we can choose \(z_n \in B_{r_n}(z_0)\cap A\).  Clearly we have
\[ |z_n - z_0 | < \frac{1}{n}, \]
for all \(n \in \mathbb{N}\), which implies \(\lim_{n\to\infty}z_n = z_0\).
\end{proof}


\begin{mydef}
\begin{enumerate}
\item The set \(A\subset \mathbb{C}\) is called \textbf{disconnected}, if there are two open sets \(U,V\) in \(\mathbb{C}\) such that
\begin{enumerate}
\item \((A\cap U) \neq \emptyset\) and \((A\cap V) \neq \emptyset\).
\item \(U\) and \(V\) are disjoint in \(A\): \((A\cap U) \cap( A\cap  V) = \emptyset\)
\item \(A\subseteq U\cup V\).
\end{enumerate}

\item The set \(A\) is called \textbf{connected} if it is not disconnected.
\end{enumerate}
\end{mydef}

\begin{prop}
The set \(A\) is connected if and only if only the sets \(A\) and \(\emptyset\) are the only sets that are both, open and closed in \(A\).
\end{prop}
\begin{proof}
(\(\Rightarrow\)) Let \(A\) be connected and suppose on the contrary that there is a set \(B\subset A\), that is both, open and closed, and different from \(A\) and \(\emptyset\).
Then \(B\) and \(A\setminus B\) are open sets in \(A\). Then, there are open sets \(U,V\subseteq\mathbb{C}\) such that \(B = A\cap U\) and \(A\setminus B = A \cap V\).
then we have:
\begin{enumerate}
\item \((A\cap U) \neq \emptyset\) and \((A\cap V) \neq \emptyset\).
\item \((A\cap U) \cap( A\cap  V) = B \cap (A\setminus B) = \emptyset\).
\item \(A = B\cup (A\setminus B) = (A\cap U) \cup( A\cap  V) = A \cap \left( U \cup V \right)\) which implies that \(A \subseteq U\cup V\).
\end{enumerate}
These three points are a contradiction since \(A\) is assumed connected.

(\(\Leftarrow\)) Let \(A\) and \(\emptyset\) the only sets that are both, open and closed in \(A\). Suppose on the contrary that \(A\) is disconnected. Then there are two open sets in \(U,V\subset \mathbb{C}\) 
satisfying the definition above. Put \(B_1  = A\cap U\) and \(B_2 = A \cap V\). Then \(B_1\) and \(B_2\) are non-empty open sets in \(A\). Since \(B_1 \cap B_2 = \emptyset\), and \(A = B_1\cup B_2\).
Then \(A\setminus B_1 = B_2\), which implies that \(B_2\) is also closed. Moreover, since \(B_1\neq \emptyset\), \(B_2 \neq A\), which is a contradiction. Then \(A\) is connected.
\end{proof}

\begin{mydef}
Let \(A\subset \mathbb{C}\). By a curve \(\gamma\) in \(A\) we mean a continuous map \(\gamma:[0,1]\to A\). We say that \(A\) is called \textbf{arc-connected} or \textbf{path-connected} if for any
two points \(z_1, z_2 \in \mathbb{C}\) there exists a curve \(\gamma\) in \(A\) such that \(\gamma(0) = z_1\) and \(\gamma(1) = z_2\).
\end{mydef}


\begin{thm}
Let \(A\subset \mathbb{C}\).
\begin{enumerate}
\item If \(A\) is arc-connected then \(A\) is connected.
\item If \(A\) is open and connected then \(A\) is arc-connected.
\end{enumerate}
\end{thm}

Here is an example of a connected set which fails to be arc-connected \href{https://en.wikipedia.org/wiki/Topologist\%27s\_sine\_curve}{Example}.

A \textbf{domain} \(\Omega \subseteq \mathbb{C}\) is an open connected set.


\begin{mydef}
Let \(A\subseteq \mathbb{C}\). An open cover of \(A\) is a family  \(\{U_{\alpha}\}_{\alpha \in I}\) of open subsets of \(\mathbb{C}\), such that \(A\subset \cup_{\alpha\in I} U_{\alpha}\). 

A finite subcover is a collection \(\{ U_{\alpha_1},\ldots, U_{\alpha_k}\}\) such that \(A \subseteq \cup_{j=1}^k U_{\alpha_j}\).
\end{mydef}

\begin{mydef}
The set \(K\subset \mathbb{C}\) is called \textbf{compact} if any open cover of \(K\), has a finite subcover.
\end{mydef}

\begin{thm}
(\textbf{Heine-Borel}). The set \(K\subset \mathbb{C}\) is compact if and only \(K\) is closed and bounded.
\end{thm}
\begin{proof}


(\(\Rightarrow\)).

If \(K\) is compact then \(K\) is closed. Supose there is \(z_0\in \mbox{cl}(K)\) such that \(z_0\notin K\), and consider the following sets: All balls \(B_r(z_0)\) and for each \(z\in K\) choose a ball \(U_z:=B_{r_z}(z)\) such that \(r_z>0\) is small enough to not intersect one of the \(B_r(z_0)\). Note now that the collection \(\{U_z \}_{z\in K}\) is an open cover of \(K\) but clearly any finite choice of \(U_z\)'s fails to cover \(K\).

If \(K\) is compact then \(K\) is bounded. For each \$z \(\in\) \(K\) considet the set \(U_{z} = B_{1}(z)\). Clearly \(\{U_z \}_{z\in K}\) is an open cover of \(K\), and then there is a finite open subcover \(U_{z_1}, \ldots, U_{z_n}\). Take \(M = \max\{|z_1| + 1, \ldots, |z_k| +1 \}\). Then \(K \subset \cup_{k=1}^{n} U_{z_k} \subseteq B_M\), i.e. \(K\) is bounded.

(\(\Leftarrow\)). Exercise.
\emph{Hint}. Show first that any closed subset of a compact set is also compact. Use the bounded property to show that \(K\) must be contained in a square of the form \(G= [-r,r]\times [-r,r]\) and show that show that \(G\) is compact.
\end{proof}

\begin{thm}
\(K\subset \mathbb{C}\) is compact if and only if for every sequence \(\{z_n\}_{n=1}^{\infty}\subset K\) has a convergent subsequence, i.e., \(\{z_{n_k}\}_{k=1}^{\infty}\) such that \(\lim_{k\to\infty} z_{n_k} = z_0\) for some \(z_0\in K\).
\end{thm}
\begin{proof}
(\(\Rightarrow\)) Let \(K\subset\mathbb{C}\) compact. If \(\{z_n\}_{n=1}^{\infty}\subset K\) is a bounded sequence since \(K\) is bounded. By Bolzano-Weierstrass' Theorem, the sequence has a convergent subsequece, say to a limit point \(z_0\in \mbox{Cl}(K)\). Since \(K\) is closed then \(z_0\in K\).


(\(\Leftarrow\)) Suppose that any  \(\{z_n\}_{n=1}^{\infty}\subset K\)  has a convergent subsequece. IF \(K\) is not bounded, then we can construct a sequence \(\{z_n\}_{n=1}^{\infty}\) such that \(|z_n| > n\) for all \(n\in \mathbb{N}\). But this sequence has no convergent subsequence, which is a contratiction. Then \(K\) should be bounded.

Now take any \(z_0\in \mbox{Cl}(K)\). By a previous proposition, we can construct a sequence  \(\{z_{n}\}_{n=1}^{\infty}\) such that \(\lim_{n\to\infty} z_n = z_0\). By our hypotesis, the sequence \(\{z_{n_k}\}_{k=1}^{\infty}\) such that \(\lim_{k\to\infty} z_{n_k} = w_0\), for some \(w_0\in K\). By the uniqueness of the limit, \(z_0 \in K\). We he have shown \(\mbox{Cl}(K) \subseteq K\), which implies \(\mbox{Cl}(K) = K\), i.e., \(K\) is closed.
\end{proof}

\begin{prop}
Let \(f:A\to\mathbb{C}\) be a function. Then the following sentences are equivalent
\begin{itemize}
\item \(f\) is continuous in \(A\).
\item For every open set \(V \subset \mathbb{C}\), the set \(f^{-1}(V)\) is open in \(A\).
\item For every closed set \(G \subset \mathbb{C}\), the set \(f^{-1}(G)\) is closed in \(A\).
\end{itemize}
\end{prop}


\begin{prop}
Let \(K\subset\mathbb{C}\) be compact and \(f: K \to \mathbb{C}\) continuous. Then \(f(K)\) is also compact.
\end{prop}


\begin{prop}
Let \(A\subset\mathbb{C}\) be connected and \(f: A \to \mathbb{C}\) continuous. Then \(f(A)\) is also connected.
\end{prop}

\begin{prop}
Let \(K\subset \mathbb{C}\) be a compact set, \(f:K\to\mathbb{C}\) continuous. Then \(f\) is uniformly continuous.
\end{prop}
\end{document}